\documentclass{article}

\usepackage{amsmath}
\usepackage{ntheorem}
\newtheorem{question}{Question}
\newcommand{\itp}{\textsc{Itp}}
\newcommand{\Tr}{\mathit{Tr}}
\newcommand{\Init}{\mathit{Init}}
\newcommand{\setof}[1]{\{#1\}}

\newcommand{\false}{\bot}
\newcommand{\true}{\top}

\begin{document}
\section{Generating monotone interpolation sequence}\label{monotone_itpseq}

Let $\Tr(V_0, V_1)$ be a transition relation, $P(V)$ a property, and $\setof{I_i}_{i=0}^k$ be
an interpolation sequence satisfying
\begin{equation}
  \forall i < k \cdot I_i(V_0) \land \Tr(V_0, V_1) \implies I_{i+1}(V_1)
\end{equation}
Define a sequence $\setof{J_i}_{i=0}^{k}$ as follows:
\begin{align}
  J_0 &= I_0\\
  J_{i+1} &= \itp (J_i(V_1) \lor J_i(V_0) \land \Tr(V_0,V_1),
  \neg I_{i+1}(V_1) \land \neg J_i(V_1))
\end{align}
where $\itp(A,B)$ is an interpolant between $A$ and $B$. Then, the
sequence $\setof{J_i}_{i=0}^k$ satisfies:
\begin{align}
  \forall i < k \cdot I_0(V_0) \land \Tr(V_0, V_1) &\implies
  I_{i+1}(V_1)\\
  \forall i < k \cdot J_i(V_0) &\implies J_{i+1}(V_0)
\end{align}

The proof is based on the following equivalences:
\begin{align}
  J_i &\implies I_i \lor J_{i-1}\\
  I_i(V_0) \land \Tr(V_0,V_1) &\implies I_{i+1}(V_1)\\
  J_{i-1}(V_0) \land \Tr(V_0,V_1) &\implies J_i(V_1)
\end{align}

Note that if $\itp(A,B)$ produces CNF, then the sequence $\setof{J_i}$ is in
CNF.

\begin{question} Is it possible to construct the sequence $\setof{J_i}$
  directly from the resolution proof of unsatisfiability of
  $\Init(V_0)\land\bigwedge_{i=0}^{k-1} \Tr(V_i,V_{i+1})\land\neg P(V_k)$.
\end{question}

\subsection{Using an Existing Sequence}

Let $\bar{F}^k = \langle F_0, F_1,\ldots, F_k\rangle$ be a monotone sequence of propositional formulas in CNF satisfying the following:
\begin{itemize}
	\item $F_0 = \Init$
	\item $F_i\Rightarrow P$ for every $0\leq i\leq k$
	\item $F_i \Rightarrow F_{i+1}$ for every $0\leq i < k$
	\item $F_i(V_0)\land\Tr(V_0,V_1)\Rightarrow F_{i+1}(V_1)$
\end{itemize}

Let us now assume that for $n > k$, the formula,
\begin{equation}
  \label{eq:unsat_fmla_one}
	F_0(V_0)\land\bigwedge_{i=0}^{k-1}[ \Tr(V_i,V_{i+1})\land F_{i+1}(V_{i+1})]\land\bigwedge_{i=k}^{n-1} \Tr(V_i,V_{i+1})\land\neg P(V_n)
\end{equation}
is unsatisfiable. Now let $\bar{I}^n = \langle I_i \rangle_{i=0}^n$ be
an interpolation sequence w.r.t.~\eqref{eq:unsat_fmla_one}. Then, the
following holds:
\begin{itemize}
	\item $I_0 = \Init$
	\item $\forall i < k \cdot I_i(V_0) \land \Tr(V_0, V_1) \implies I_{i+1}(V_1)$
	\item $I_n\Rightarrow P$
\end{itemize}

Our goal now is to make $\bar{I}^n$ both monotone and in CNF. To do so, there are two possibilities. Either use the formulation presented above, or use a slightly revised version of it making use of the already existing $\bar{F}^k$ sequence:

\begin{itemize}
	\item $J_0 = I_0$
	\item $\forall i < k \cdot J_{i+1} = \itp (A,B)\land F_{i+1}$ where \\$A = J_i(V_1) \lor J_i(V_0) \land \Tr(V_0,V_1)$ and $B = \neg I_{i+1}(V_1) \land \neg J_i(V_1)\land\neg F_{i+1}(V_1)$
	\item $\forall (k\leq i < n) \cdot J_{i+1} = \itp (A,B)$ where \\$A = J_i(V_1) \lor J_i(V_0) \land \Tr(V_0,V_1)$ and $B = \neg I_{i+1}(V_1) \land \neg J_i(V_1)$
\end{itemize}

One should note that using $n > k+1$ pose a problem. Let us denote $\bar{J}^n = \langle J_i \rangle_{i=0}^n$ as the resulting monotone interpolation sequence in CNF. While each of the first $k$ elements satisfy $P$, the rest (except for the last one), do not satisfy $P$. To remedy this we slightly change the formulation and simply conjoin $\neg P$ in the $B$ part. We then get that $\forall (k\leq i < n) \cdot J_{i+1} = \itp (A,B)$ where $A$ is the same as before and $B = \neg I_{i+1}(V_1) \land \neg J_i(V_1)\land\neg P(V_1)$. The addition of $\neg P$ is possible since it only further restricts the formula (therefore maintains its unsatisfiability). In addition, it also further restricts the resulting interpolant and therefore does not break any of our desired properties.

Question / Issues / Comments:
\begin{enumerate}
	\item Note that for $0\leq i < k$, we still conjoin $F_{i+1}$ to the resulting interpolant, even though it does not change the cardinality of the formula (by cardinality I mean "strength" - cardinality= the number of satisfying assignments, it is logically equivalent). The question is then: do we need it? Considering our implementation, this will be done in the SAT solver in case we keep the same instance throughout the run (incremental instance). But in the actual representation of our sequence (for fixpoint purposes for example), there is no need in maintaining the extra clauses. Moreover, it may even be better to get rid of these clauses (both in our maintained sequence and in the SAT-solver).

	\item What are the implications on the Justification process? Meaning, in the original $\bar{F}^k$ the clauses were justified. In the new $\bar{F}^n$ (in case we discard the previous $\bar{F}^k$, none of the clauses is justified. Do we justify them all? Do we want to keep some of the $\bar{F}^k$ clauses and replace some new clauses by these to save Justification operations?
\end{enumerate}

\subsection{Using an Existing Sequence, Take 2}

Assume everything as in the previous section. However, instead of
computing the interpolants from~\eqref{eq:unsat_fmla_one}, consider
the following equivalent sequence partitioning:
\begin{equation}
  \label{eq:unsat_fmla_two}
  \bigwedge_{i=0}^{k}[F_i(V_i) \land \Tr(V_i,V_{i+1})] \land
  \bigwedge_{i=k+1}^{n-1} \Tr(V_i,V_{i+1})\land\neg P(V_n)
\end{equation}
That is, $F_i$ is pushed to the $B$-part of each interpolant instead
of the $A$-part. Let $\bar{I}^n = \langle I_i \rangle_{i=0}^n$ be an
interpolation sequence w.r.t.~\eqref{eq:unsat_fmla_two}. Then, the
following holds:
\begin{align}
  F_0 \land I_0 &= \Init\\
  \forall i < k \cdot F_i(V_0) \land I_i(V_0) \land \Tr(V_0, V_1)
  &\implies F_{i+1}(V_1) \land I_{i+1}(V_1)\\
  F_n \land I_n &\implies P
\end{align}
where for all $i > k$, $F_i = true$. That is, the sequence $\bar{I}$
is the strengthening of the sequence $\bar{F}$ sufficient to show that
the new unrolling is unsatisfiable. If $\bar{F}$ and $\bar{I}$ are
both in CNF, then $I_i$ is the set of clauses that are added to $F_i$
to strengthen it. The sequence $\bar{I}$ can be put into CNF and made
monotone relative to~\eqref{eq:unsat_fmla_two} as usual. For
justification, the sequence $\bar{F}$ can be kept constant or
not. This is left as a choice for now. Note that if $\bar{F}$ is
constant, then the SAT solver's context is grown monotonically (more
constraints are added) for each subsequent unrolling. Hence, all
intermediate results (i.e., learned clauses) can be kept.

\section{CNF-ization}

In this section we use a notion similar to the one appearing in the beginning of Section~\ref{monotone_itpseq}. Recall that adding monotonicity to an interpolation-sequence involves the following:

\begin{align}
  J_0 &= I_0\\
  J_{i+1} &= \itp (J_i(V_1) \lor J_i(V_0) \land \Tr(V_0,V_1),
  \neg I_{i+1}(V_1) \land \neg J_i(V_1))
\end{align}

We would like to not only add monotonicity to the resulting sequence but also get a sequence in CNF. To do so, we can use IC3-like method. Let us consider iteration $i$ of the monotonicity process. We know that the formula $(J_i(V_1) \lor J_i(V_0) \land \Tr(V_0,V_1)) \land \neg I_{i+1}(V_1) \land \neg J_i(V_1)$ is unsatisfiable. We therefore define $J_{i+1}$ to be the constant $\true$. Following our partitioning to $A$ and $B$ we get that $A\Rightarrow J_{i+1}$. The missing part is that $J_{i+1}\land B$ must be unsatisfiable. We start by finding an assignment $s$ (over $V_1$) of $J_{i+1}\land B$. Clearly, since $A\land B$ is unsatisfiable, $A\land s$ is unsatisfiable. Also, due to the satisfiability of $s\land B$ we know that $J_i\Rightarrow\neg s$ (and so is every $J_j$ for $j < i$ since these are already monotone). Due to that we can apply IC3-like inductive generalization by solving the following unsatisfiable formula:
\begin{align}
  J_i(V_0)\land\neg s(V_0) \land \Tr(V_0,V_1)\land s(V_1)
\end{align}

Similarly to IC3, $\neg s$ is a clause and $s$ is a cube. The result of the inductive generalization process is a sub-clause $c\sqsubseteq \neg s$ such that the following holds:
\begin{align}
  J_i(V_0)\land c(V_0) \land \Tr(V_0,V_1)\Rightarrow c(V_1)
\end{align}

This clause is then added to the computed interpolant $J_{i+1}$ and to every interpolant $J_j$ for $j < i+1$. The resulting sequence is then not only monotone, but also syntactically contains the same clauses (like in IC3).

If on top of that we add the computation of clauses generated by the SAT solver (CAV'13 paper), we no longer have an (syntactic) inclusion of clauses.

There are several interesting aspects to the above:
\begin{enumerate}
	\item Without using the CAV'13 SAT solver, we get similar properties to the sets computed by IC3. But our method lets the SAT solver roam all over the unrolled formula (unlike IC3 and alike - latest FMCAD paper by Alan Hu and friends).
	\item Using the CAV'13 approach to create the CNF interpolants give a different angle to the algorithm. Note that in this case Justification plays a bigger role in eliminating unnecessary clauses. Where in the IC3-oriented approach Justification is different. Note sure that "bigger role" is the right term.
\end{enumerate}


\section{Justification}

\end{document}
%%% Local Variables:
%%% mode: latex
%%% TeX-master: t
%%% End:
